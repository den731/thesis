The increase in harmful cyanobacteria blooms threatens freshwater ecosystems and presents a risk to human health. The rapid growth of cyanobacteria can cause rapid declines in water quality if left unchecked. A survey of 29 inland lakes was conducted to investigate microcystin. Liquid chromatography-mass spectrometry (LC-MS/MS) was used to identify and quantify the microcystin variants. Measurement of nutrient concentrations and water quality parameters were conducted. With Geographic Information System (GIS), land-use for each lake's watershed was calculated. In addition, Quanitative Polymerase Chain Reaction (QPCR) was used to detect and quantify the 16s ribosomal RNA (\emph{16s rRNA}), \emph{mcyE}, \emph{cyrA} and \emph{sxtA} gene targets responsible for microcystin and nodularin, cylindrospermopsin and saxitoxin.  I hypothesized total microcystin would be positivly associated with lake's with urbanized watershed. In our survey, total microcystin did not correlate with urbanization. 
