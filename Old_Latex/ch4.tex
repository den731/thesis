


Using a predictive model can potentially be a vital utility for protecting the public from HABs. Protecting public health is a balance of scientific knowledge, risk assessment of the situation and maximizing available tools at our disposal. It is apparent that it is difficult to model such a complex dynamic. In general there are no best models in an ideal sense, but statistical modeling can help identify meaningful relationships. Point sampling does not encapsulate all the complex relationship and does not explain subtle changes. Confounding variables such as unaccounted disturbances not measured from our survey is not accounted. 

We are currently exploring artificial neural-networking to model our full dataset.
