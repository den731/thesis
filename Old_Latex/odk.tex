\section*{Purpose}

\indent Data aggregation and structural collection is one of the few key factors that determines the quality of the study. Large scale studies are often done by a team of multiple groups and organizations. There are available commercial tools which can aid in data collection and aggregation automatically. ArcGIS \footnote{ESRI 2017. ArcGIS Desktop: Release 10.5.1 Redlands, CA: Environmental Systems Research Institute.} coupled with their Android/IOS app called Collector for ArcGIS \footnote{https://www.esri.com/products/collector-for-arcgis}.

From my experience I found a viable alternative which does not require an expensive license. I wish to help other research studies with a cheaper alternative and greatly facilitate their pursuit. I have written this guide to allow anyone to build from the ground up on setting a database and providing a viable way to organize an enterprise. This guide is aimed to initially setup a database for non-profit organization and academic research with a limited budget. The initial cost is \$5 per month for a virtual private server (VPS) which has enough storage for a small database. I explored a the tool from Open Data Kit \footnote{Open Data Kit 2.0: Expanding and Refining Information Services for Developing Regions \url{http://www.hotmobile.org/2013/papers/full/2.pdf} Waylon Brunette, Mitchell Sundt, Nicola Dell, Rohit Chaudhri, Nathan Breit, Gaetano Borriello In HotMobile, 2013. \url{http://dl.acm.org/citation.cfm?id=2444790}} and has served me in other projects with limited resources  in great success. The real beneficial aspect of this tool is the ability to scale and work with other API's for extra functionality. With the VPS, you can scale up your server for more storage and CPU power when more data is being collected as the project grows. Open ODK is open-source and is being implemented to work with other common application with their use of API's.  I discovered a way to utilize this tool and had my teammates collect data using their android device. The alternative would be laborious, involving me to compile multiple data which does introduce more probability of error and prone to more mistakes.



\section*{Requirements}

\indent Here you must setup a server that stores you data and be accessible through the Internet. Having your own computer machine as a server is a viable solution but it requires more extensive networking knowledge. An alternative solution is to purchase a rent a server from a third party which automatically sets up and quickly gets connected to the Internet. A Virtual Private Server (VPS) are cheap and flexible servers to set up a database up on the cloud. For as little as \$5 dollars a month one can have a server with everything they need to start collecting data.

DigitalOcean, Amazon (free trial for a year), Google VPS can work. I suggest anyone of them, but in this guide we will choose to setup from DigitalOcean as its the cheapest available server. I recommend Debian or Ubuntu linux distribution to setup your database. In this guide, we will choose Debian 9 (Stretch).


\noindent
You must get an SSH client to access the server for setting up ODK Aggregate. For Linux or Mac, use the native terminal application to access the server. The command is simply:

\begin{lstlisting}[language=bash]
  $  ssh root@xxx.xxx.xxx.xxx # replace xxx.xxx.xxx.xxx with the given IP address sent by the provider.
\end{lstlisting}


\noindent
For windows users, I recommended program is PuTTy \url{https://www.putty.org/}. Install this program for windows. Open up PuTTy and enter in the IP address and click ``Open''. You need to know what the IP address is for your server, the user name and password. Usually this is emailed to you from the provider.



\section*{Setup}

\noindent
From a fresh installation, the first prompt when accessing the server will be to change your password. Choose a good strong password for the root username and have it written down.  After setting the password, preform any upgrades and install sudo which allows a normal user to have elevated privileges and install nano, a text editor. This may be already installed but some distributions might not have them. To do this, follow these commands.

\begin{lstlisting}[language=bash]
  $  apt update # Update the repository
  $  apt upgrade -y # Install any outdated packages
  $  apt dist-upgrade -y # Install distribution wide updates
  $  apt install sudo nano # Install sudo and nano
\end{lstlisting}

\noindent
Next, we will set a separate username. The \emph{root} user is only used to administer the server.  Choose any name as you see fit and something you can remember. For our example, for the case of this guide, \emph{dummy} will be the username.

\begin{lstlisting}[language=bash]
  $  adduser dummy
\end{lstlisting}

\noindent
You will be prompted to create a password. When asked for name and other user information, hit ``Enter'' to leave it blank. This does not need to be filled out.


\noindent Here, we will also assign this new user as a sudo user. This will enable the new user to have elevated privelages.

\begin{lstlisting}[language=bash]
  $  usermod -a -G sudo dummy # Assign the user dummy to sudo group
\end{lstlisting}




\section*{Secure Server}



\noindent
Its highly advisable to setup a pubic/private RSA key pair for more secure login setup. This prevents unauthorized access to the server and prevents the basic brute-force attacks. Here are the steps neccessary to secure the server against malicious attacks. We will also setup a firewall called ufw (Uncomplicated Firewall) which is very simple to setup and secure your server. We will also generate a key-pair for this user. You can have a passphrase which is an additional layer of security, or leave it blank. When generated, the public key will be located at \path{/home/dummy/.ssh/id_rsa.pub}. Follow this command:

\begin{lstlisting}[language=bash]
  $  su dummy # Switch to dummy user
  $  sudo apt install ufw # Install ufw
  $  sudo ufw allow 22 # Allow SSH and SCP
  $  sudo ufw allow 8080 # Allow ODK Aggregate port
  $  sudo ufw allow 5432 # Allow PostgreSQL access
  $  ssh-keygen # Generate RSA key-pair

\end{lstlisting}

\noindent
Next, we will move the generate public key to the authorized key folder. We will also limit the permissions of the authorized keys for only the user dummy to access, install putty tools and use it to convert the private key for windows client to use:

\begin{lstlisting}[language=bash]
  $  sudo mv ~/.ssh/id_rsa.pub ~/.ssh/authorized_keys # Copy the public key
  $  chmod 600 ~/.ssh/authorized_keys # Modify the permission
  $  sudo apt install putty-tools # Install putty tools
  $  cd ~/.ssh # Change directory to ssh folder
  $  puttygen id_rsa -o id_rsa.ppk # Convert the key to PuTTy readable file

\end{lstlisting}

\noindent
Next, you have to retrieve the converted key file from the server to your own local machine. With windows, I recommend using WinSCP for secure file transfer \url{https://winscp.net/eng/index.php}. When prompted, enter the IP address in host name, port 22, and enter the new username and password (dummy). In WinSCP, hidden files are not shown by default. Hit Ctrl+Alt+H to show hidden files. On the right side panel, you will find the servers directory. The left side is the local computer's directory. Here we can transfer files between the machines securely. On the right side, locate .ssh folder and find the id\_rsa.ppk file and download it to your document folder. Simply double-click the file to download it to your local machine.

\noindent
This file can be pre-loaded int PuTTy. Double click the id\_rsa.ppk file, open up PuTTy and type the IP address. Entering in the username will automatically log in without password as the key authenticates you.

\noindent
Next, we will disable login by password which only allows users with the key file to access the server. This secures the server as it prevents malicious bots from brute-force attacks:

\begin{lstlisting}[language=bash]
  $  sudo nano /etc/ssh/sshd_config
\end{lstlisting}

\noindent
The command will open up \emph{nano} text editor. Scroll down the config file by pressing the down arrow key. Find the line that contains

\begin{lstlisting}
  >$    PermitRootLogin yes
\end{lstlisting}

\noindent
Change it to:

\begin{lstlisting}[language=bash]
  >$    PermitRootLogin no
\end{lstlisting}

\noindent
Scroll all the way to the end and find the line containing:

\begin{lstlisting}[language=bash]
  >$    PasswordAuthentication yes
\end{lstlisting}

\noindent
And change it to:

\begin{lstlisting}[language=bash]
  >$    PasswordAuthenticion no
\end{lstlisting}

\noindent
Make sure there is no \# in the beginning of the changed lines. Exit nano text editor by pressing ``Ctrl+X''. Press ``Y'' to save file and hit ``Enter'' to write file.

\noindent
Next restart ssh:

\begin{lstlisting}[language=bash]
  $  sudo service ssh restart
\end{lstlisting}

\section*{Install ODK Aggregate}

\noindent
ODK Aggregate is a java web applet that handles the data and stores into a database. The web applet is hosted as a website and can be accessed through any web browser. On the host machine, we will need to setup tomcat server that hosts ODK Aggregate along with an SQL server. PostgreSQL is used as we can also extend this to work with spatial data and be accessible from QGIS. As of writing this guide, ODK Aggregate is version 1.5.0. You may need to find the current version. Go to \url{https://github.com/opendatakit/aggregate}.

\begin{lstlisting}[language=bash, breaklines=true]
  $  sudo apt install tomcat8 unzip # This install tomcat version 8 and unzip to allow us to extract .zip files
  $  wget https://github.com/opendatakit/aggregate/releases/download/v1.5.0/ODK-Aggregate-v1.5.0-Linux-x64.run.zip
  $  unzip ODK-Aggregate-v1.5.0-Linux-x64.run.zip
  $  chmod +x ODK-Aggregate-v1.5.0-Linux-x64.run # Allow the file to be executable
  $  ./ODK-Aggregate-v1.5.0-Linux-x64.run  # Execute the .run file
\end{lstlisting}

\noindent
You will see a prompt when the ODK Aggregate Setup Wizard appears. Follow these steps:

\begin{enumerate}


\item Read the license. Hit ``Enter'' to scroll through the document. Type ``Y'' at the end of the document to accept the license terms.

\item Type \path{ODK} to create a folder where the necessary files will be created.


\item  Type ``3'' to setup a PostgreSQL platform setup

\item Type ``Y'' to downloaded
\item Type ``N'' for ssl
\item Type ``1'' for no ssl
\item Type ``Y'' for port config
\item Hit ``Enter'' to default 8080
\item Type the IP address of the server.
\item Hit ``Y'' for PostgreSQL
\item Hit ``Enter'' to set the default PostgreSQL port to 5432
\item Hit ``Enter'' for default username \emph{odk\_user}
\item Make up a password for this user login
\item It will ask the name of the database \emph{odk\_prod}
\item Then it will ask to setup a name for your schema. \emph{odk\_prod}

\item The ``ODK Aggregate Instance Name'' is used as a display for your users. This can typically be set for a project name or title. For our purpose, it will be set to
\emph{demo}.
\item Create a super user name, most likely whomever is managing the data. We will set ours to be called \emph{super}.

\end{enumerate}

\noindent
Next, you must setup PostgreSQL:
\begin{lstlisting}[language=bash]
  $  echo deb http://apt.postgresql.org/pub/repos/apt/stretch-pgdg main | sudo tee -a /etc/apt/sources.list
  $  wget --quiet -O https://www.postgresql.org/media/keys/ACCC4CF8.asc | sudo apt-key add -
  $  sudo apt update
  $  sudo apt install postgresql-9.4
\end{lstlisting}

Next you need to create postgres as a username and setup a password. Here you will be switching into \emph{psql} command prompt to set things up. Once you are done, you will exit \emph{psql} by typing $\backslash$q which will lead into the regular bash home directory. Follow these command:

\begin{lstlisting}[language=bash]
  $  sudo -u postgres psql postgres
    % postgres=# \password postgres
    % postgres=# \q
  $
\end{lstlisting}

\noindent
Now you must set up the configuration of PostgreSQL to allow remote connections. You will edit the \path{/etc/postgreql/9.4/main/pg_hba.conf} file and the \path{/etc/postgresql/main/postgresql.conf}:

\begin{lstlisting}[language=bash]
  $  sudo nano /etc/postgresql/9.4/main/pg_hba.conf
\end{lstlisting}

\noindent
Change the line that has this:
\begin{lstlisting}[language=bash]
  >$    local   all    all      peer
\end{lstlisting}
\noindent
To this:
\begin{lstlisting}[language=bash]
  >$    local   all    all      md5
\end{lstlisting}

\noindent
Also add this line at the very end of the file:
\begin{lstlisting}[language=bash]
  >$    host   all     all      0.0.0.0/0 md5
\end{lstlisting}

\noindent
Then exit by pressing ``Ctrl+X'', then hit ``Y'' to confirm, then hit ``Enter''. Lets edit the other file:

\begin{lstlisting}[language=bash]
  $  sudo nano /etc/postgresql/9.4/main/postgresql.conf
\end{lstlisting}

\noindent
Find this line:
\begin{lstlisting}[language=bash]
  >$    #listen_addresses = 'localhost'
\end{lstlisting}

\noindent
Replace it with this, note the removal of  \# at the beginning of the line:

\begin{lstlisting}[language=bash]
  >$    listen_addresses = '*'
\end{lstlisting}


\noindent
Now you will run the SQL configuration file generated from ODK Aggregate to setup the database in PostgreSQL. This will create the user \emph{odk\_user} and the database and schema \emph{odk\_prod}. Once this is done, you will restart PostgreSQL and have the database online.

\begin{lstlisting}[language=bash]
  $  sudo -u postgres psql postgres
      % postgres=# \cd '/home/dummy/ODK/ODK\ Aggregate'
      % postgres=# \i create_db_and_user.sql
      % postgres=# \q
  $  sudo systemctl restart postgresql
\end{lstlisting}


\noindent
ODK Aggregate run file from the previous step generated a .war file that needs to be copied to apache tomcat webapps folder. With tomcat8 installed on debian, the webapps folder is located at \path{/var/lib/tomcat8/webapps}.


\begin{lstlisting}[language=bash]
  $ cd $HOME/ODK/ODK\ Aggregate/ && sudo cp ODKAggregate.war /var/lib/tomcat8/webapps
  $  sudo service apache restart
\end{lstlisting}

\noindent
Now you should have the ODK Aggregate server running. You can simply access the server by going to your web browser on your local computer and access it by replacing your given ip address of the server in this form: \url{http://xxx.xxx.xxx.xxx:8080/ODKAggregate} Log into your super user login and click ``Site Admin''. Here you can add more users to administer the site or be a data collector. Site administrators have the ability to create more users and have all other power as a data collector. For a simple user who is just collecting data, you can have them set as ``Data Collector'' and not be able to change any data that has already been collected. Next, we must create a form for the data collectors to use.


\section*{Survey Forms}

Before you start collecting data, you must set up a predefined form which consists of variables of interest. The ODK Collect app can collect spatial data in a form of GPS points, path of coordinates or trace a spatial polygon. ODK Aggregate can accept .xml files which needs to be created. The ODK Suite provides an online tool to create a desired form \url{https://build.opendatakit.org/}. When you first go to the site, you will be asked to sign up for a login. You do not have to create an account and can simply click cancel.

Here you can design your survey form. On the top-right hand corner, you must name this form. This will be displayed for the user, so choose something that pertains to what kind of data is being collected. Next, you will choose what kind of data are going to be collected. The most important are usually the date/time and GPS coordinates. On the bottom, there is an option to select location. When selected, a configuration window will appear on your right-hand side. For the name, name it location. You can have it configured to be a single point, a path which creates a polyline, or a shape which creates a polygon shape. For the date and time, you can configure the full date and time or just the date. Depending on your project, you can select a variety of data ranging from audio or video data or a design questionnaire. For audio or video there is a media option. You can collect audio, video and picture or have all three options if you wish. The questionnaires can be designed by clicking the select multiple option. Here you can have options for the data collector to choose from. This can be configured to have a follow-up questions depending on what options the data collector initially selects. More information can be found by clicking help for more advance configurations.

Once you have a satisfied form, you must save the form as an .xml file. On the top-right hand side, click File and select Export to XML. Once saved, head over you your ODK Aggregate website and click on ``Form Management''. Click ``Add New Form'' and select the .xml file to upload to the server.

\section*{ODK Collect App}

Next, the data collectors need to have the ODK Collect app to start collecting data. Android users can download the ODK Collect app either through the Google Play store or directly download the .apk file from \url{https://github.com/opendatakit/collect/releases/tag/v1.15.1}. Unfortunately there is no app for the IOS platform.

First, you must set the settings. Go to the ``general settings'' and tap on ``server'' settings. For the option ``Type'' choose the \emph{ODK Aggregate} option. The URL will be the the same as the ODK Aggregate web address. Username and password will be the users that are created from ODK Aggregate website, or the super user. Head back to the main menu and click on ``Get Blank Form''. This will download the forms you have uploaded to the ODK Aggregate server. The app can collect data offline and send the filled forms once connected online. For each entry of data, tap the ``Fill Blank Form'' option and select the appropriate form. Here you will start filling out the form. Once completed, you will save the form offline. If you are connected to the Internet, you will have to upload the filled forms to the ODK Aggregate server. To do so, head to the main menu and select the ``Send Finalized Form'' option and simply tap on the form that is saved. This will now upload the data to your server.

\section*{Conclusion}

The utility from using ODK tools is beneficial for large scale studies, especially handing with geospatial data. ODK Aggregate is very flexible in working with other forms of applications. With PostgreSQL database, the stored data can be accessed either throught the ODK Aggregate webserver or by connecting to the PostgreSQL database connection. QGIS ,ArcGIS Desktop or Microsoft Access can connect to this database and pull data. For more advance configuration, see the ODK documentation: \url{https://docs.opendatakit.org/}. You may want to consider registering a domain name which allows you to enter a human-recognizable address instead of an IP address.

There are a few caveats with this implementation. First, there is no application to input data from a device other than Android. However, there are implementation of a web form which can be accessed through a computers web browser or even IOS and submit forms using OpenRosa \footnote{https://enketo.org/openrosa/}
