\chapter{CONCLUSION}

\section{Environmental Drivers}

Using a predictive model is a vital utility for protecting the public from \gls{hab}. Addressing public health is a balance of scientific knowledge, risk assessment of the situation and maximizing available tools at our disposal. It is apparent that it is difficult to model \gls{hab} as it involves more complex dynamic. In general there are no best models in an ideal sense, but statistical modeling can help identify meaningful relationships. Point sampling does not encapsulate all the complex relationship and does not explain subtle changes. Confounding variables such as unaccounted disturbances not measured from our survey is not accounted.

Its important to understand why \gls{hab} is occurring. There are other factors to consider that our study did not observe. Urbanization also have effect with releases of heavy metals such as copper, lead, iron and zinc \cite{clausen_introduction_2018}


Hydrodynamic Modeling

\section{SPATT}

