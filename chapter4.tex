% !TEX root = main.tex
\chapter{CONCLUSION}

In our next year of survey we will test the hypothesis if the biofilm has a negative effect of microcystin adsorption.

Its important to understand why \gls{hab} is occurring and is worth to investigate further. Using a predictive model is a vital utility for protecting the public from \gls{hab}. Addressing public health is a balance of scientific knowledge, risk assessment of the situation and maximizing available tools at our disposal. It is difficult to realistically model \gls{hab} as it involves more complex dynamic. In general there are no best models in an ideal sense, but statistical modeling can help identify meaningful relationships. Point sampling does not encapsulate all the complex relationship and does not explain subtle changes. Confounding variables such as unaccounted disturbances not measured from our survey is not accounted. Some studies built predictive models based using cyanobacterial cell counts, biomass and species composition as a response variable as its most likely associated with \gls{hab} \cite{moore_richard_cyanobacterial_1993, ahn_evaluation_2011, jiang_statistical_2008, beaulieu_nutrients_2013, taranu_predicting_2017}. Predicting \gls{mc} alone may not be the most ideal approach, but still worthwhile to investigate nonetheless. In our future work, the use of \gls{qpcr} to measure mRNA or tRNA for gene expression may provide more indepth of what environmental parameters could influence the production of cyanotoxins. The use of \gls{qpcr} has shown to be effective in predicting \gls{hab} \cite{wilson_genetic_2005}. However there is some valid criticism against the use of \gls{qpcr}. First, the quantified concentrations of genes are not distinguishable between live or dead cells. There are also limitations of relating gene copies to cell counts or concentrations of cyanotoxins \cite{pacheco_is_2016}. Nevertheless, the use of \gls{qpcr} can be still helpful but should be used in parellel to \gls{lcmsms} and cell counts.






