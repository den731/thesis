% !TEX root = main.tex
\chapter{CONCLUSION}

From our analyses, we found ELISA and \gls{lcmsms} to have good agreement except in the month of October where ELISA generally reported high values. We believe this could be due to the cross reactivity with non MCs, particularly degradation products common in the fall. We also observed some discrepancies with the SPATT compared to our grab samples. This could have been due to where some lakes had substantial biofilm encasing the SPATT bags, in particular to lakes with \gls{hab}, preventing the adsorbtion of MC. In our next year of survey we will test the hypothesis if the biofilm has a negative effect of MC adsorption.

It is difficult to realistically model \gls{hab} as it involves with dynamic relationships between abiotic and biotic factors, cyanobacteria growth and cyanotoxin production. We identified orthophosphate to be associated with \gls{mc}, but not neccessarily with cyanobacteria(in terms of gene copies). We also found turbidity which can be a surrogate parameter of other factors to be also a good corrlate. However, turbidity can be caused from cyanobacterial cells. In general there are no best models in an ideal sense, but statistical modeling can help identify meaningful relationships. Point sampling does not encapsulate all the complex relationship and does not explain subtle changes. Confounding variables such as unaccounted disturbances not measured from our survey is not accounted. Even with the use of SPATT, there are still some challenges to overcome. Predicting \gls{mc} alone may not be the most ideal approach, but still worthwhile to investigate nonetheless. Other previous studies built predictive models based using cyanobacterial cell counts, biomass and species composition as a response variable as its most likely associated with \gls{hab} \cite{moore_richard_cyanobacterial_1993, ahn_evaluation_2011, jiang_statistical_2008, beaulieu_nutrients_2013, taranu_predicting_2017}. As of writing this thesis, we are still working on identifying and the cyanobacteria. 

In our future work, the use of \gls{qpcr} to measure mRNA or tRNA for gene expression may provide more indepth of what environmental parameters could influence the production of cyanotoxins. The use of \gls{qpcr} has shown to be effective in predicting \gls{hab} \cite{wilson_genetic_2005}. However there is some valid criticism against the use of \gls{qpcr}. First, the quantified concentrations of genes are not distinguishable between live or dead cells. There are also limitations of relating gene copies to cell counts or concentrations of cyanotoxins \cite{pacheco_is_2016}. Nevertheless, the use of \gls{qpcr} can be still helpful but should be used in parallel to \gls{lcmsms} and cell counts.

Its important to understand why \gls{hab} is occurring and is worth to investigate further. Using a predictive model is a vital utility for protecting the public from \gls{hab}. Addressing public health is a balance of scientific knowledge, risk assessment of the situation and maximizing available tools at our disposal. 






